\section{Funkcje szumu}

W zależności od potrzeby, ale konsekwentnie w całym tekście, nazwiska podaje się najczęściej w jeden z trzech sposobów:
\begin{itemize}
	\item poprzedzone pełnym imieniem,
	\item poprzedzone inicjałami imion,
	\item bez dodatków.
\end{itemize}
Od tej zasady mogą wystąpić wyjątki, np. w pierwszym miejscu wystąpienia cytowania poprzedza się go pełnymi imionami, albo inicjałami imion, a we wszystkich następnych podaje się tylko nazwisko. Niezależnie od przyjętej zasady dla całej pracy, nazwiska ogólnie znane cytuje się często bez imion i inicjałów (np. Kopernik, Einstein w pracach fizycznych). Jeżeli jedynymi występującymi w tekście nazwiskami są nazwiska autorów wymienionych w bibliografii, z~reguły nie poprzedza się ich imionami lub inicjałami imion.

Przytaczanie w tekście nazwisk i imion obcojęzycznych podaje się je w całej pracy w sposób jednolity \pauza w~oryginalnej postaci, a nie spolszczonej (chyba, że jest u nas tradycyjnie przyjęta). Często, aby poprawnie zapisać imię lub nazwisko, potrzebna jest wiedza jak je wymawiać. Przykładowo, od tego czy samogłoska \textit{e} na końcu jest wymawiana, czy jest niema, zależy dołączana polska końcówka (będzie więc w~dopełniaczu Verne'a i~Scharnkego). Należy znać również zasadę, że odmieniając imię lub nazwisko kończące się spółgłoską lub samogłoską \textit{y}, nie wstawia się apostrofu przed polską końcówką (np. Grallem, Halla, Barneyowi).