\section{Algorytm Marching Cubes}

Pisząc pracę dyplomową autor ma do czynienia z różnego rodzaju elementami takimi jak: pojęcia i definicje, twierdzenia, dowody, wnioski, a także wszelkie opisy zjawisk i rzeczy. Aby podać je w jasnej postaci, która będzie zrozumiała dla czytelnika, należy posłużyć się równocześnie wieloma środkami wyrazu.

Niestety, nawet najbogatsza w słowa polszczyzna często może nie wystarczyć aby w pracy dyplomowej mającej charakter naukowy wyłożyć właściwe treści w sposób maksymalnie jednoznaczny. Dodatkowo używa się takich środków językowych jak: wyrazy i zwroty, które są metodycznym narzędziem służącym do nadania treści odpowiedniej postaci, czyli \textbf{językowy aparat pomocniczy}, oraz wyrazy i zwroty ściśle określonym specyficznym znaczeniu, czyli wszelkie \textbf{terminy specjalistyczne} z różnych dziedzin nauki i techniki. Należy pamiętać, że podczas pisania pracy bogactwo językowe często jest wadą a nie zaletą, a zbędne słowotwórstwo jest wręcz zakazane. Należy pisać językiem prostym, a jednocześnie naukowym.



W odniesieniu do wyrazów i zwrotów zarówno tworzących językowy aparat pomocniczy, jak i wchodzących w skład terminów specjalistycznych, należy przestrzegać symetrii językowej w antonimach, tj. parach wyrazów o przeciwnym znaczeniu. Nie można równocześnie pisać  \textit{absolutny} i~\textit{względny},  \textit{dodatni} i~\textit{negatywny},  \textit{aktywny} i~\textit{bierny} itp. Tego rodzaju parom należy nadać symetryczną postać, używając ogólnie przyjętych wyrazów języka polskiego, a nie obcego. Podane wcześniej jako przykład antonimy powinny wyglądać tak:  \textit{bezwzględny} i~\textit{względny},  \textit{dodatni} i~\textit{ujemny},  \textit{czynny} i~\textit{bierny}.

Tej ostatniej zasady należy przestrzegać w odniesieniu do wszystkich wyrazów i~zwrotów używanych w tekście pracy. Zamiast np. \textit{produkt} lepiej pisać \textit{wyrób}, a~zamiast \textit{produkt kartezjański} lepiej \textit{iloczyn kartezjański}, zamiast \textit{realizacja} \pauza \textit{wykonanie}, zamiast \textit{realizowany} \pauza \textit{wykonany}, zamiast \textit{kompatybilny} \pauza \textit{zgodny}, zamiast \textit{kalkulacje} \pauza \textit{obliczenia}, zamiast \textit{relewantny} \pauza \textit{istotny} itp.


Ważną kwestią jest również używanie polskich słów zamiast obcojęzycznych oryginalnych odpowiedników. Warto zapoznać się z opublikowanymi pracami i sprawdzić czy dane pojęcie nie zostało już po polsku nazwane. Przykładowo, w matematyce istnieją czysto polskie wyrazy \textit{całka} i \textit{różniczka}.