\section{Opis zagadnienia i przykłady zastosowania}

Najważniejszą częścią pracy dyplomowej, zaraz po wykonaniu części praktycznej, jest jej tekst główny. Zasady przygotowania tekstu opracowano na podstawie normy~\cite{Norma}.

Podstawowy podział tekstu głównego to rozdziały i podrozdziały, którym należy przypisać numery od najmniejszego do największego. Od tej zasady można odstąpić w niektórych przypadkach, jednak zalecane jest aby tego nie robić, ponieważ może to spowodować problemy z czytelnością struktury pracy. Wspomniana wcześniej norma~\cite{Norma} dopuszcza dwa sposoby oznaczania poszczególnych części tekstu:
\begin{itemize}
\item \textbf{numeracja liczbowa--wielorzędowa}, w której do oznaczania wszystkich partii tekstu używa się tylko liczb arabskich, powtarzając w oznaczeniach partii niższego stopnia oznaczenia partii wyższego stopnia,
\item \textbf{numeracja liczbowo--literowa} (mieszana), w której hierarchię tekstu określa się stosując (w takiej a nie innej kolejności) następujące oznaczenia: liczby rzymskie \pauza litery wielkie \pauza liczby arabskie \pauza litery małe.
\end{itemize}
Zaleca się jednak aby dla tekstów z dziedziny nauk ścisłych i techniki stosować pierwszy z wymienionych sposobów. Należy mieć na uwadze, że tekst może wymagać aby być podzielonym na rozdziały i podrozdziały. Jednocześnie należy pamiętać, że rozdrabnianie tekstu i wprowadzanie szczegółowego podziału jest zbędne. 

Przykładową numerację rozdziałów i podrozdziałów zastosowano w niniejszym szablonie, dzięki czemu nie trzeba martwić się jej budowaniem, a wystarczy skupić uwagę na samej logicznej strukturze pracy i odpowiednim podziale tekstu na rozdziały.

Przy numeracji wielorzędowej obowiązuje jeszcze jedna niepisana zasada: między tytułem wyższego stopnia a tytułem pierwszej partii tekstu niższego stopnia nie powinno być żadnych tekstów. W praktyce tekst taki nazywa się wiszącym. Zgodnie z~tą zasadą np. po tytule rozdziału~1 powinien od razu następować tytuł rozdziału~1.1, a tuż po tytule rozdziału~1.5 (o ile istnieje kolejny poziom podrozdziałów) \pauza tytuł podrozdziału~1.5.1, itp.

Inną formą podziału, które nie są umieszczane w obrębie spisu treści są wyliczenia. Należy pamiętać o konsekwentności stosowania wyliczeń (przynajmniej w~obrębie jednego rozdziału), czyli stosować jeden wybrany sposób dla całej pracy. Znacząco poprawi to jej wygląd. Przykładowo stosując małe litery alfabetu mamy listę
\begin{enumerate}[(a)]
	\item pierwszy element,
	\item drugi element,
	\item trzeci element.
\end{enumerate}
Oczywiście można stosować wyliczenia wielostopniowe, np.
\begin{enumerate}
	\item pierwszy element,
		\begin{itemize}
			\item pierwszy element z wielostopniowym podziałem,
			\item i kolejny,
		\end{itemize}
	\item drugi element.
\end{enumerate}
Jednak ze względów edytorskich nadmiernie stosowanie tego rodzaju wyliczeń nie jest mile widziane. Oznaczenie wyliczenia jest formą dowolną \pauza można stosować gwiazdki, trójkąty, kropki, myślniki, liczby arabskie czy litery alfabetu, jednak należy pamiętać o tym, aby jedno oznaczenie pojawiało się w obrębie całej pracy. Wyjątkiem od tej reguły jest sytuacja gdy zróżnicowane oznaczenia oznaczają hierarchię w wyliczeniach lub podkreślają odmienny charakter niektórych wyliczeń.

Ostatnim omawianym elementem podziału jest akapit, czyli rozbicie tekstu na małe fragmenty uwarunkowane logicznym układem. Tekst dzieli się na akapity zgodnie z przyjętą zasadą: nowa myśl \pauza nowy akapit. Są jednak pewne \textit{ale} natury formalnej. Akapity nie powinny być za krótkie, oraz zbyt długie, ponieważ gęsto zbity tekst nie ułatwia czytania. 

Kończąc pisać pracę należy zwrócić uwagę na logiczną analizę podziału tekstu. Należy odpowiedzieć sobie na następujące pytania takie jak:
\begin{itemize}
	\item Czy tytuł pracy odpowiada treści tekstu głównego? 
	\item Czy układ tekstu głównego jest logiczny?
	\item Czy podział tekstu głównego na części, części na rozdziały, a rozdziały na podrozdziały jest logiczny?
	\item Czy tytuły rozdziałów i podrozdziałów odpowiadają zawartym w nim treściom?
	\item Czy tekst główny nie jest obarczony nadmiarem uwag ogólnych, powszechnie znanych wiadomości o dziedzinie, której dotyczy praca, zawartych we wprowadzeniach do poszczególnych rozdziałów i podrozdziałów?
	\item Czy w tekście głównym nie występuje nadmiar powtórzeń? Przykładowo, czy we wprowadzeniu do rozdziałów i podrozdziałów nie opisuje się tego o czym będzie nieco dalej, a w podsumowaniach nie powtarza tego o czym była już mowa?
	\item Czy w tekście głównym autor nie zamieścił szczegółowych opisów ilustracji, bądź tekstów, które powinny się znaleźć w podpisach pod ilustracjami?
	\item Czy pewnych treści uzupełniających nie należy przenieść do dodatków?
\end{itemize} 