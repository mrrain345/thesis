\section{System generowania biomów}

Umieszczając elementy graficzne należy wziąć pod uwagę m.in. ich zawartość treściową, poprawność merytoryczną, kompletność zamieszczanych danych, a także poprawność użytych symboli graficznych, oznaczeń literowych, itd. Wszelkie ilustracje umieszczone w pracy powinny być wykonane w odpowiedniej skali, tak aby były zawsze czytelne. Dane oraz informacje zawarte w ilustracjach również powinny być odpowiednio dobrane pod względem ilościowym oraz jakościowym, a także odpowiednio rozmieszczone aby nie wpływać na czytelność ilustracji. Zaleca się aby wszystkie ilustracje umieszczane w pracy (o ile jest to możliwe) były tego samego formatu. Wpływa to na jakość oraz porządek opracowania. Ilustracje powinny być umieszczane w miejscu tekstu gdzie jest o nich mowa, a więc na tej samej stronie lub na stronie poprzedzającej lub następne, ale tak, żeby je można było oglądać łącznie z dotyczącym ich tekstem. Każda ilustracja powinna posiadać numer. Numeracja powinna być ciągła w obrębie  całego opracowania lub dwurzędowa i zgodna z numeracją tabel wzorów itd. Podpisy pod 
ilustracjami powinny być umieszczane pod ilustracją. Wszelkie objaśnienia powinny być umieszczone 
zaraz pod podpisem do ilustracji.