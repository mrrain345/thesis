\section{System dynamicznego generowania terenu}

Często w opracowaniach zachodzi konieczność umieszczania różnego rodzaju zdjęć, wykresów, diagramów, algorytmów itp. różniących się jakością, rozmiarem oraz formatem zapisu grafiki. Różnice w elementach graficznych wynikające głownie z formatu zapisu mogę znacząco wpłynąć na jakość pracy. Korzystając z pakietu \LaTeX\ zaleca się, aby grafika umieszczana w opracowaniu zapisana była w formacie \textit{pdf} lub \textit{png}. Najwyższą jakość uzyskuję się zapisując elementy graficzne w formacie \textit{pdf}. Stosując ten format podczas skalowania elementu jego jakość graficzna będzie zawsze ta sama. Oczywiście pakiet \LaTeX\ wspiera również inne formaty zapisu, tj. \textit{jpg}, \textit{bmp}, itd.