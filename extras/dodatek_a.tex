\chapter{Cel dodatków w pracy}
%=================================================================================================

Do materiałów, które mogą uzupełnić pracę, oprócz rysunków, tablic, ilustracji itp. należą także dodatki, nazwane również aneksami. Dają one możliwość albo dołączenia do tekstu głównego różnorodnego rodzaju informacji dodatkowych, albo wyłączenia z tekstu głównego tych wiadomości, które nie są w nim konieczne. Niekiedy pewne wiadomości wplecione w tekst niepotrzebnie go obciążają, przerywają zasadniczy wątek lub są nadmiernie szczegółowe. Jeśli mimo to wiadomości te są użyteczne i mogą być przydatne, warto oczyścić z nich tekst główny i zgrupować je na końcu pracy w postaci dodatków.


Przykładem użycia dodatków może być opis zawartości płyty CD lub DVD dołączonej do pracy lub instrukcje laboratoryjne stworzone w oparciu o napisaną pracę.

%=================================================================================================