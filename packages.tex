% ------------------------------------------------------------------------
% pakiet do wzorów ams
% ------------------------------------------------------------------------
\usepackage{amsmath}
\usepackage{amssymb}

% ------------------------------------------------------------------------
% język polski
% ------------------------------------------------------------------------
\usepackage[MeX]{polski}
\usepackage[utf8]{inputenc}

% ------------------------------------------------------------------------
% obsługa pdf
% ------------------------------------------------------------------------
\usepackage[pdftex,usenames,dvipsnames]{color}	%obsługa kolorów

% ------------------------------------------------------------------------
% wstawienie danych o autorze i pracy 
% ------------------------------------------------------------------------
\usepackage[pdftex,
				pagebackref=false,						% referencje w spisie literatury do strony na, której została urzyta
				draft=false,								% draft
				pdfpagelabels=false,						%
				pdfstartview=FitV,						% lub FitH
				pdfstartpage=1,							% 
				bookmarks=true,							% zakładki w pliku pdf
				pdfauthor={Autor},						% należy wpisać autora pracy
				pdftitle={Praca inżynierska},			% 
				pdfsubject={Tytuł pracy},				% tytuł
				pdfkeywords={slowa kluczowe},			% słowa kluczowe
				unicode=true]{hyperref}   


% ------------------------------------------------------------------------
%	style
% ------------------------------------------------------------------------
\usepackage{extsizes}							%więcej rozmiarów czcionek
\usepackage[a4paper,left=3.5cm,right=2.5cm,top=2.5cm,bottom=2.5cm]{geometry}
\usepackage{tocloft}								% format spisu treści
\usepackage{array}								% lepiej wyglądające tabelki
\usepackage[format=hang,
				labelsep=period,
				labelfont={bf,small},
				textfont=small]{caption}		% formatuje podpisy pod rysunkami i tabelami
\usepackage{floatflt}							% ładniejsze opisywanie obrazków tekstem
\usepackage{subfig}								% możliwość wstawiania figur w kolumnach
\usepackage{graphicx}							% do obsługi grafiki
\usepackage{here}									% wymuszanie położenia figury w danym miejscu
\usepackage{url}									% adresy internetowe
\usepackage{enumerate}							% modyfikowanie list wyliczeniowych np \begin{enumerate}[(a)]...
\usepackage{multirow}							% do tabel 
% ------------------------------------------------------------------------
% listingi
% ------------------------------------------------------------------------
\usepackage{listings}							% do wstawiania listingów programów




\usepackage{slantsc} % Pochyłe kapitaliki  np. \textsl{\textsc{Automatyka i robotyka}}

% ------------------------------------------------------------------------
% inne
% ------------------------------------------------------------------------
\usepackage{glossaries}

\usepackage{dashrule}
\usepackage{fancyhdr} 							% do stopki i nagłówka
\usepackage{calc}
\usepackage{packages/zmienne}					% zmienne dodatkowe używane min. w karcie pracy oświadczeniu i stronie tytułowej zebrane w jednym miejscu
\usepackage{packages/strona_tytulowa}
\usepackage{packages/oswiadczenie}
\usepackage{packages/karta_pracy}
\usepackage{packages/pusta_strona}
\usepackage{packages/wspolrealizacja}
\usepackage{longtable}							% do podziału tabel na wiele stron

\usepackage{indentfirst}

% ------------------------------------------------------------------------
% kodowanie czcionek
% ------------------------------------------------------------------------
\usepackage[T1]{fontenc}
\usepackage{lmodern}\normalfont %to load T1lmr.fd 

% ------------------------------------------------------------------------
% do algorytmów
% ------------------------------------------------------------------------

\usepackage{algorithm}
\usepackage{algorithmic}
\floatname{algorithm}{Algorytm}

% ------------------------------------------------------------------------
% do nomenklatury
% ------------------------------------------------------------------------

\usepackage[section]{placeins}
\usepackage{nomencl}
\makenomenclature
\usepackage{makeidx}
\makeindex
\renewcommand{\nomname}{Spis ważniejszych symboli}
% na końcu pliku z nomenklaturą należy umieścić polecenie
% \printnomenclature
% plik należy dodać poleceniem \input
%% przykład treści pliku
%% \nomenclature{$\oplus$}{Dylatacja zbioru}
%% \printnomenclature


