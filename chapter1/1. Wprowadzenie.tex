\section{Wprowadzenie}

Niniejszy przykład stanowi propozycję układu treści pracy pisemnej inżynierskiej lub magisterskiej, oraz krótkie przedstawienie jak należy prawidłowo pisać pracę pod względem edytorskim.

Zdecydowano się na pokazanie możliwości jakie w tym zakresie daje system składu tekstu \LaTeX. Ułatwia on edycję tekstu pozwalając autorowi skupić się na treści i~strukturze tekstu. Dzięki temu autor nie musi koncentrować się na szczegółach technicznych (np. formatowaniu danych bibliograficznych, tabel, podpisów pod rysunkami, standardach numerowania wzorów i nagłówków, itp.).

Początkowo praca z tekstem pisanym w \LaTeX 'u może przynieść sporo problemów oraz wymaga nauczenia się zestawu niezbędnych poleceń, jednak już po niewielkim czasie można zauważyć efekty, szczególnie jeżeli chodzi o poprawne wstawianie wzorów, tabel czy rysunków. Trzeba również zauważyć, że język opisu równań matematycznych jest tak wygodny w użyciu, że stosuje się go niejednokrotnie w serwisach internetowych takich jak Wikipedia.

Niniejszy tekst ten oparto głównie o pozycję \cite{Osuchowska:1998}, w której poruszono wiele kwestii związanych z poprawnym edytowaniem tekstu, jak również o pozycję \cite{latex:2007}, w której opisano zagadnienia związane z notacją \LaTeX 'a.

Opracowanie może służyć jako wzór do napisania własnej pracy dyplomowej lub magisterskiej. Zostało przemyślane w taki sposób aby skupić uwagę autora na tworzeniu tekstu zamiast zastanawianiu się czy dobrze go sformatowano. Podstawowa wiedza na temat oprogramowania \LaTeX\ powinna wystarczyć do tego aby napisana praca miała przejrzysty wygląd i dobrze się prezentowała. 
