\section{Cel i zakres pracy}

W tym miejscu należy jasno sformułować cel pracy. Podrozdział ten pisze się tak, aby czytelnik mógł zrozumieć motywację podjęcia badań bez konieczności zapoznania się z dodatkową literaturą. Należy uprzednio dokonać przeglądu literaturowego, który ma pokazać, że autor ma świadomość historii badań nad rozwiniętym problemem oraz aktualnego ich stanu. Powinno się tu wskazać na korelacje z istniejącymi podejściami oraz brakiem istniejącej wiedzy motywujące podjęcie badań objętych pracą dyplomową.
Przykład: Celem pracy było opracowanie i realizacja projektu prezentacji dydaktycznej, przeznaczonej dla osób pragnących uzupełnić swoją wiedzę w~zakresie poznania języka \LaTeX\ oraz zasad składu tekstu.

Praca swym zakresem obejmowała:
\begin{itemize}

\item zgromadzenie i zapoznanie się z literaturą tematu,
\item opracowanie założeń projektu,
\item zaprojektowanie struktury logicznej pracy,
\item realizację projektu oraz usunięcie błędów.
\end{itemize}