\chapter{Rendering terenu}

\input{chapter3/1. Rekalkulacja wektorów normalnych}
\section{Mapowanie tekstur}

Przed przystąpieniem do tworzenia tabeli należy się zastanowić czy umieszczone w tabeli dane nie są powieleniem danych w tekście głównym. Ponowne umieszczanie tych samych danych w tabelach powoduje niepotrzebne zwiększenie objętości pracy, co pogarsza jej walory. Jednocześnie tekst, który zawiera dużą ilość liczb traci walory narracyjne, przez co należy się zastanowić czy nie warto umieścić danych liczbowych w tabelach. W tabelach obowiązuje nazewnictwo identyczne jak w tekście głównym. Informacje, które mają zostać umieszczone w tabelach, należy dobrać pod względem ilościowym oraz jakościowym. Nie zaleca się przytaczania z obcych źródeł obszernych zestawień, bez uprzedniej selekcji. Jeśli praca wymaga przytoczenia takich zestawień, należy się zastanowić nad sensem umieszczania całej treści zestawienia. Często wystarczy wymienić kilka, najwyżej kilkanaście zamieszczonych w~nich pozycji. Powtarzanie w~tabelach tych samych danych, które umieszczone są na ilustracjach, jest błędem. W tym przypadku należy zdecydować się na jeden sposób reprezentowania danych.


Każda tabela zaczerpnięta z obcego źródła musi posiadać odpowiednią adnotację o tym źródle. Jeśli tabela zawiera wyniki własne i cudze, też powinno się to oznaczyć. Informacji o źródle można nie podawać, gdy wszystkie tabele pochodzą od autora dzieła. Tytuł tabeli powinien być możliwie krótki i zwięźle sformułowany. Wyjątek stanowią niewielkie tabela będące częścią tekstu głównego, które często nazw nie mają. Wszelkie oznaczenia oraz skróty wymagają objaśnienia bezpośrednio pod tytułem tabeli, albo u dołu tabeli. Objaśnień nie wolno podawać w tekście głównym. Jeśli się tam znajdują należy przenieść je do tabeli. Powtarzające się oznaczenia w~kolejnych tabelach można umieścić w pierwszej tabeli, a w kolejnych tabelach podać odnośnik do miejsca oznaczenia np. \textit{Oznaczenia jak w tab. \ref{tab:zlewiska}}. W tabelach nie wolno zostawiać pustych kratek. W przypadku gdy pojawiają się puste kratki, należy umieścić odpowiedni znak umowny, tj.:
\begin{itemize}
\item -- (kreska) \pauza zjawisko nie występuje;
\item 0 (zero) \pauza zjawisko istnieje jednak w ilościach mniejszych od liczb, które mogą być podane w tabeli;
\item . (kropka) \pauza zupełny brak informacji lub brak wiarygodnych informacji;
\item znak $\times$ \pauza wypełnienie rubryki ze względu na układ tabeli jest niemożliwy lub niecelowy.
\end{itemize}  

%=================================================================================================
% \section{System dynamicznego generowania terenu}
% Tabela składa się z wierszy i kolumn. Pozycją w tabeli nadajemy wspólne nazwy, wprowadzając odpowiednie nagłówki kolumn i wierszy. Wiersze numeruje się tylko wówczas, gdy numeracja jest rzeczywiście potrzebna. Zwyczajowo kolumna z numeracją oznaczona jest jako \textit{Lp.}. Najczęściej oznaczenie takie jest 
% zbędne. Kolumny numeruje się tylko w bardzo długich zestawieniach, gdy tabela zajmuje kilka lub kilkanaście stron. Jeżeli tablica jest bardzo duża i nie mieści się na jednej stronie, należy na kolejnej stronie powtórzyć numer oraz tytuł tablicy dodając w nawiasach \textit{ciąg dalszy} lub \textit{cd}. Nagłówki kolumn powinny być powtórzone na kolejnych stronach.
% %=================================================================================================
% \section{System generowania biomów}
% Poniżej podano kilka przykładów tabel które mogą być wzorem do opracowywania własnych 
% zestawień.
% \begin{table}[h!]
% 	\centering
% 	\caption{Wartości parametrów określających zależność $\mu(N,E)$ dla krzemu.}
% 	\begin{tabular}{|c|c|c|c|c|c|c|}
% 		\hline  
% 		\multirow{2}{*}{Rodzaj nośników} & \multicolumn{6}{c|}{Parametry} \\\cline{2-7}
% 				& $\mu_0\,[\mathrm{m}^2/\mathrm{V\cdot s}]$ & $N_0\,[\mathrm{m}^{-3}]$ & $\mathrm{S}$ & $A\,[\mathrm{V}/\mathrm{m}]$& $E_0$ & $B\,[\mathrm{V}/\mathrm{m}]$\\
% 		\hline
% 			Elektrony & $1.4\cdot 10^{-1}$ & $3\cdot 10^{22}$ & 350 & $3.5\cdot 10^5$ & $8.8$ & $7.4\cdot 10^5$ \\
% 		\hline
% 			Dziury & $4.8\cdot 10^{-2}$ & $4\cdot 10^{22}$ & 81 & $6.1\cdot 10^5$ & $1.6$ & $2.4\cdot 10^6$ \\
% 		\hline
% 	\end{tabular}
% \label{tab:wartosciKrzemu}
% \end{table}
% %---------------------------------------------------------------------------------------------------
% \begin{table}[h!]
% 	\caption{Zestawienie wybranych członów dynamicznych.}
% \centering
% 	\begin{tabular}{|c|c|c|}
% 		\hline
% 			Nazw członu & Równanie różniczkowe & Transmitancja operatorowa G(s)\\
% 		\hline
% 			\multirow{2}{*}{Proporcjonalny} & \multirow{2}{*}{$y(t)=kx(t)$} & \multirow{2}{*}{$k$}\\
% 			  &  & \\
% 		\hline
% 			\multirow{2}{*}{Różniczkujący idealny} & 
% 				\multirow{2}{*}{$y=k\dfrac{\mathrm{d}x(t)}{\mathrm{d}t}$} 
% 				& \multirow{2}{*}{$ks$}\\
% 				&  & \\
% 		\hline
% 			\multirow{2}{*}{Różniczkujący rzeczywisty} & 
% 				\multirow{2}{*}{$T\dfrac{\mathrm{d}y}{\mathrm{d}t}+y(t)=Tk\dfrac{\mathrm{d}x(t)}{\mathrm{d}t} $} 
% 				& \multirow{2}{*}{$\dfrac{Tks}{Ts+1}$}\\
% 				&  & \\
% 		\hline
% 		   \multirow{2}{*}{Całkujący} & \multirow{2}{*}{$\dfrac{\mathrm{d}y(t)}{\mathrm{d}t}=kx$} & 
%          \multirow{2}{*}{$\dfrac{k}{s}$}\\
% 				&  & \\
% 		\hline
% 			\multirow{2}{*}{Inercyjny I rzędu} & \multirow{2}{*}{$T\dfrac{\mathrm{d}y(t)}{\mathrm{d}t}=kx(t) $} 
% 				& \multirow{2}{*}{$\dfrac{k}{Ts+1}$}\\
% 				&  & \\
% 		\hline
% 	\end{tabular}
% \label{tab:zestTrans}
% \end{table}
% \newpage
% %---------------------------------------------------------------------------------------------------
% \begin{table}[h!]
% \caption{Pochodne funkcji elementarnych na przykładzie podstawowych funkcji trygonometrycznych.}
% \centering
% 	\begin{tabular}{c p{3cm} c p{3cm} c p{3cm}}
% 		\hline
% 			Funkcja & & Pochodna & & Uwagi\\
% 		\hline
% 			$\sin(x)$ & & $\cos(x)$ & & --\\
% 			$\cos(x)$ & & $-\sin(x)$& & --\\ 
% 			$\tg(x)$ & & $\dfrac{1}{\cos^2(x)}$& & $x \neq \dfrac{\pi}{2}+k\pi,\; k\in \mathbb Z$ \\
% 			$\ctg(x)$ & & $-\dfrac{1}{\sin^2(x)}$ & &$x \neq k\pi,\; k\in \mathbb Z$\\
% 		\hline
% 	\end{tabular}
% \label{tab:zlewiska}
% \end{table}


















