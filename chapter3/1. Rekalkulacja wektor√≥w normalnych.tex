\section{Rekalkulacja wektorów normalnych}

Podczas pisania pracy dyplomowej niejednokrotnie pojawia się problem przedstawienia czytelnikowi w jednolity i zrozumiały sposób danych, tj. danych liczbowych (np. wartości funkcji liczbowych), danych statystycznych,
właściwości fizycznych, chemicznych, itp. Wszelkie dane (nie tylko te, które zostały wymienione) łatwiej analizować, porównywać itd. jeśli zostaną przedstawione w postaci tabeli. Aby jednak tabela prezentowała dane w pożądany sposób należy, podobnie \pauza jak w każdym przypadku pisania pracy \pauza trzymać się określonych zasad.